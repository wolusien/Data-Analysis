\documentclass[french]{article}
\usepackage[T1]{fontenc}
\usepackage[utf8]{inputenc}
\usepackage{lmodern}
\usepackage[a4paper]{geometry}
\usepackage{babel}
\usepackage{bm}
\usepackage{enumitem}
\usepackage{xcolor}
\usepackage{amsmath} 
\usepackage{amssymb}  
\usepackage{gensymb}
\newcommand*{\norme}[1]{\left\lVert{#1}\right\rVert}

\definecolor{blue}{RGB}{51,131,255}
\title{Vocabulaire anglais}
\author{DJEBALI Wissam}
\begin{document}
	\maketitle
	
\begin{itemize}[label=\textbullet]
	\item \textbf{nested} = niché
	\item \textbf{cumbersome} = encombrant 
	\item \textbf{alleviate} = soulager
	\item \textbf{readily} = facilement
	\item \textbf{sift} = tamiser
	\item \textbf{overwhelming} = écrasant, accablant, irrésistible, insurmontable
	\item \textbf{partly} = en partie, partiellement
	\item \textbf{boundary} = limite
	\item \textbf{surrogate} = substitut, remplaçant
	\item \textbf{crude} = brut, grossier
	\item \textbf{Recall} = \textbf{Sensitivity}
	\item \textbf{False-Discovery Rate FDR} = \textbf{1-Precision}
	\item \textbf{goodness-of-fit} = qualité de l'ajustement
	\item \textbf{overfitting} = surraprentissage
	\item \textbf{sparsity} = parcimonieux, rare
	\item \textbf{be plagued} = en proie
	\item \textbf{Voronoi tesselation} = pavage de Voronoi
	\item \textbf{tile} = tuile
	\item \textbf{non overlapping} = non chevauchant
	\item \textbf{stringent} = rigoureux, sévère, strict
	\item \textbf{handful} = poignée, quarteron
	\item \textbf{wiggly} = sinueux
	\item \textbf{mild} = doux, léger, faible
	\item \textbf{palatable} = appétent, savoureux
	\item \textbf{entail} = entraîner, comporter
	\item \textbf{recipe} = recette
	\item \textbf{batch} = lot
	\item  \textbf{least squares} = méthodes des moindres carrés
	\item \textbf{cross-entropy} = entropie croisée \textbf{(voir log-vraisemblance)}
	\item \textbf{conversely} = inversement, réciproquement
	\item \textbf{curse of dimensionality} = malédiction de la dimensionnalité : Plus on a de variables plus on a besoin d'avoir un échantillon de grande taille pour assurer la pertinence ou la significativité de l’information que l’on peut lire des données. 
	\item \textbf{hone} = affiner, aiguiser, affûter
	\item \textbf{binning} = \textbf{bucketing} = regrouper en groupes des valeurs
	\item \textbf{etangled} = enchevêtré
	\item \textbf{depth} = profondeur
	\item \textbf{bears consideration} = mérite qu'on s'y attarde
	\item \textbf{manifold} = multiple, multitude
	\item \textbf{embed} = intégrer, encastrer
	\item \textbf{knot} = noeud
	\item \textbf{singular matrix} = matrice non inversible(déterminant égale à 0)	\\
\end{itemize}

\section{Vocabulaire Topologie}
\begin{itemize}[label=\textbullet, font=\LARGE \color{blue}]
	\item \textbf{manifold} = variété : Une \textbf{variété de dimension} $\bm{n}$, où $\bm{n}$ désigne un entier naturel, est un espace topologique localement euclidien, c'est-à-dire dans lequel tout point appartient à une région qui s'apparente à un tel espace. \\
	
	\item \textbf{homeomorphism} = homéomorphisme : En topologie, un homéomorphisme est une application bijective continue, d'un espace topologique dans un autre, dont la bijection réciproque est continue. Dans ce cas, les deux espaces topologiques sont dits homéomorphes.\\\\	
	\textbf{La notion d'homéomorphisme est la bonne notion pour dire que deux espaces topologiques sont « le même » vu différemment. C'est la raison pour laquelle les homéomorphismes sont les isomorphismes de la catégorie des espaces topologiques}.\\
	
	\item \textbf{homotopy} = homotopie :
	L'homotopie est une notion de topologie algébrique. Elle formalise la notion de déformation continue d'un objet à un autre. Deux lacets sont dits homotopes lorsqu'il est possible de passer continument de l'un à l'autre. Ce concept se généralise à bien d'autres objets que des lacets.\\
	
	\item \textbf{isotopy} = isotopie : L’isotopie est un raffinement de l'homotopie ; dans le cas où les deux applications continues $\bm{f,g : X \rightarrow Y}$ sont des \textbf{\textit{homéomorphismes}}, on peut vouloir passer de $\bm{f}$ à $\bm{g}$ non seulement continûment mais en plus par homéomorphismes.\\
	
	On dira donc que $\bm{f}$ et $\bm{g}$ sont isotopes s’il existe une application continue $\bm{H : X \times [0, 1] \rightarrow Y}$ telle que :
	\begin{itemize}[label=$\star$]
		\item $\bm{\forall x \in X, H(x, 0)=f(x)}$
		\item $\bm{\forall x \in X, H(x, 1)=g(x)}$
		\item \textit{pour tout }$t\in[0,1]$, l'application $\bm{X \rightarrow Y, x \mapsto H(x,t)}$ est un homéomorphisme\\
	\end{itemize}
	
	\item \textbf{\textit{ambient isotopy}} : \\ Formally, an \textbf{\textit{ambient isotopy}} between \textbf{\textit{manifolds}} $\bm{A}$ and $\bm{B}$ is a continuous function $\bm{F:[0,1]\times X\rightarrow Y}$ such that each $\bm{F_{t}}$ is a homeomorphism from $\bm{X}$ to its range, $\bm{F_{0}}$ is the identity function, and $\bm{F_{1}}$ maps $\bm{A}$ to $\bm{B}$. That is, $\bm{F_{t}}$ continuously transitions from mapping $\bm{A}$ to itself to mapping  $\bm{A}$ to $\bm{B}$.\\ \\
	\textbf{\textit{isotopie ambiante}} :\\ 
	Une variante est la notion d'\textbf{\textit{isotopie ambiante}}, qui est une sorte de déformation continue de l'« espace ambiant », transformant progressivement un sous-espace en un autre : deux plongements $\bm{\alpha}$, $\bm{\beta}$ d'un espace $\bm{Z}$ dans un espace $\bm{X}$ sont dits « isotopes de manière ambiante » s'ils se prolongent en deux homéomorphismes $\bm{f, g}$ de $\bm{X}$ dans lui-même isotopes (au sens précédent) ou, ce qui est équivalent, s'il existe une isotopie entre l'identité de $\bm{X}$ et un homéomorphisme $\bm{h}$ de $\bm{X}$ dans lui-même tel que $\bm{h \circ \alpha=\beta}$. Cette notion est importante en théorie des nœuds : deux nœuds sont dits équivalents s'ils sont reliés par une isotopie ambiante.\\
	
\end{itemize}
\end{document}
